\documentclass{beamer}

\mode<presentation>
{
  \usetheme{Warsaw}
  % or ...

  \setbeamercovered{transparent}
  % or whatever (possibly just delete it)
}


\usepackage[english]{babel}
\usepackage[utf8]{inputenc}
\usepackage{times}
\usepackage[T1]{fontenc}


\title[]
{Implementacja wydajnego wzorca wstrzykiwania zależności dla złożonych grafów zależności}

\author[Adrian Mularczyk]
{Adrian Mularczyk}


\institute[Uniwersytet Wrocławski]
{
Uniwersytet Wrocławski\\
Wydział Matematyki i Informatyki\\
Kierunek: Informatyka
}

\date{}

% Delete this, if you do not want the table of contents to pop up at
% the beginning of each subsection:
%\AtBeginSubsection[]
%{
  %\begin{frame}<beamer>{Outline}
    %\tableofcontents[currentsection,currentsubsection]
  %\end{frame}
%}

\begin{document}

\begin{frame}
  \titlepage
\end{frame}

\begin{frame}{Outline}
  \tableofcontents
\end{frame}



\section{Przedstawienie problemu}

\begin{frame}{Przedstawienie problemu}
  \begin{itemize}
  \item
  	W grafach zależności typy się powtarzają
  \item
  	Nowe obiekty są często tworzone
  \end{itemize}
\end{frame}


\subsection{SOLID}

\begin{frame}{SOLID}
\end{frame}

\begin{frame}{Dependency Inversion Principal}
\end{frame}


\subsection{Kontenery wstrzykiwania zależności}

\begin{frame}{Kontenery wstrzykiwania zależności}
\end{frame}


\section{Wstrzykiwanie zależności}

\subsection{Rodzaje wstrzykiwań zależności}

\begin{frame}{Rodzaje wstrzykiwań zależności}
\end{frame}

\begin{frame}{Wstrzykiwanie przez konstruktor}
\end{frame}

\begin{frame}{Wstrzykiwanie przez metodę}
\end{frame}

\begin{frame}{Wstrzykiwanie przez właściwość}
\end{frame}


\subsection{Implementacje przemysłowe}

\begin{frame}{Implementacje przemysłowe}
\end{frame}



\section{Implementacja}

\subsection{CIL}

\begin{frame}{CIL}
\end{frame}


\subsection{Reflection.Emit}

\begin{frame}{Reflection.Emit}
\end{frame}


\subsection{Pomysł na wykorzystanie tego}

\begin{frame}{Pomysł na wykorzystanie tego}
\end{frame}


\subsection{Dwa rozwiązania}

\begin{frame}{Dwa rozwiązania}
\end{frame}


\subsection{Partial Emit Function}

\begin{frame}{Partial Emit Function}
\end{frame}


\subsection{Full Emit Function}

\begin{frame}{Full Emit Function}
\end{frame}




\section{Wyniki}

\subsection{Testy wydajnościowe}

\begin{frame}{Testy wydajnościowe}
\end{frame}


\section*{Podsumowanie}

\begin{frame}{Podsumowanie}

\end{frame}

\end{document}


