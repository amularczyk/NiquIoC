\documentclass[12pt]{article}
\usepackage[left=3.5cm,right=2cm,top=2.5cm,bottom=2.5cm]{geometry}
\usepackage[T1]{fontenc}
\usepackage[utf8]{inputenc}
\usepackage{graphicx}
\usepackage{tgpagella}
\usepackage{enumerate}
\usepackage{url}
\usepackage{multirow}
\usepackage{longtable}
\usepackage{polski}
\usepackage{graphicx}
\usepackage{float}
\usepackage{color}
\usepackage{amsmath}
\usepackage{longtable}
\usepackage{tabularx}
\usepackage{ltablex,booktabs}
\usepackage{amssymb}
\usepackage{rotating}
\usepackage{subfigure}
\usepackage{float}
\usepackage{tabu}
\usepackage{caption}
\usepackage{footnote}
\usepackage{xspace}
\usepackage{mathpazo}
\usepackage[table]{xcolor}
\usepackage[justification=centering]{caption}

\pagestyle{empty}

\title{\LARGE{Uniwersytet Wrocławski}\\
\Large{Wydział Matematyki i Informatyki}\\
\large{Kierunek: Informatyka}}

\date{}

\begin{document}
\pagestyle{empty}

\begin{titlepage}
\maketitle
\thispagestyle{empty}


\begin{center}
\author{\LARGE{Adrian Mularczyk}}
\vspace{30pt}

\huge{\textbf{Stworzenie wydajnego wzorca wstrzykiwania zależności dla złożonych grafów zależności}}
\vspace{50pt}
\end{center}

\begin{flushright}
\large{Praca wykonana pod kierunkiem}
\large{dr. Wiktora Zychli}
\end{flushright}

\vfill
\begin{center}
\begin{large}
Wrocław, 2016
\end{large}
\end{center}
\end{titlepage}

\setlength{\parindent}{0pt}	%usunięcie wcięć
\setlength{\parskip}{1.5ex} 
\renewcommand*{\figurename}{Rys.}
\renewcommand*{\tablename}{Tab.} 
\renewcommand{\captionsize}{\small}

\clearpage

\tableofcontents

\clearpage

\section{Wstęp}
\subsection{Cel pracy}
Wstrzykiwanie zależnośc jest wzorcem projektowym, który pozwala na tworzenie kodu o luźniejszych powiązaniach, łatwiejszego w testowaniu i modyfikacji. Najbardziej popularnymi implementacjami tego wzorca w języku C\# są Autofac, StructureMap, Unity i Windsor, a najbardziej wydajnymi DryIoc, LightInject i SimplyInjector. Celem niniejszej pracy magisterskiej jest stworzenie wydajnej implementacji tego wzorca dla złożonych grafów zależności. Do tego celu zostanie wykorzystana funkcjolanosci z przestrzeni nazw Reflection.Emit. W tej pracy zostaną przedstawione dwa rozwiązania.

\clearpage

\section{Wstrzykiwanie zależnosci}
Jest to zbiór zasad projektowania oprogramowania i wzorców, które pozwalają nam rozwijać luźno powiązany kod \cite{dependency_injection} (str. 4).\\
Jakiemu celowi ma służyć wstrzykiwanie zależności? Wstrzykiwanie zależnści nie jestem celem samym w sobie, raczej jest to środek do celu. Ostatecznie celem większości technik programowania jest dostarczenie jak najwydajniej działającego oprogramowania. Jednym z aspektów tego jest napisanie utrzymywalnego kodu.\\
O ile nie pisze się prototypu lub aplikacji, które nigdy nie mają kolejnych wersji (kończą się na wersji 1), to wkrótce będzie trzeba zająć się utrzymaniem i rozwijaniem istniejącego kodu. Aby być w stanie pracować wydajnie z takim kodem bazowym, musi on być jak najlepiej utrzymywalny.\\
Wstrzykiwanie zależności jest niczym więcej niż techniką, która umożliwia luźne powiązania, a luźne powiązania sprawiają, że kod jest rozszerzalny i łatwy w utrzymaniu. \cite{dependency_injection} (str. 5)\\


\section{Implementacja}
<Wyjaśnić jak działa Reflection.Emit i do czego zostanie użyte>

\subsection{Rozwiązanie 1}
<opis PartialEmitFunction>

\subsection{Rozwiązanie 2}
<opis FullEmitFunction>

\section{Testy wydajnościowe}
<wyniki testów i ich opis>

\section{Podsumowanie}
<parę słów na koniec>

\newpage
\begin{thebibliography}{authordate1}
\bibitem{dependency_injection} Dependency Injection in .NET, Mark Seemann\\
\end{thebibliography}

\end{document}